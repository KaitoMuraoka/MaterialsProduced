\RequirePackage{plautopatch}
\RequirePackage[l2tabu, orthodox]{nag}

\documentclass[platex,dvipdfmx]{jlreq}			% for platex
% \documentclass[uplatex,dvipdfmx]{jlreq}		% for uplatex
\usepackage{graphicx}
\usepackage{bxtexlogo}

\usepackage{physics}%ブラケット
\usepackage{amsmath} % 行列表現
\usepackage{amsmath, amssymb}
\usepackage{type1cm}
\usepackage{qcircuit}

\title{量子アルゴリズムの大統一理論:量子特異値変換}

\author{9BSP1118 村岡海人}
\date{\today}
\begin{document}
\section{量子特異値変換}
任意の行列が与えられた時に任意の特異値分解ができる。

特異値とは、任意の行列$A$はユニタリー行列$W, V$を用いて、
% Aは正方行列出なくても良い。
\begin{align}
    A = W \Sigma V^{\dagger}
    \label{singular value}
\end{align}
と表される。ただし、$\Sigma$は対角行列。

量子特異値変換は、(\ref{singular value})のように分解できる行列$A$がある際に、
\begin{align}
    U = W f(\Sigma)V^{\dagger} 
\end{align}
のようなユニタリー演算子を作ることができる。量子特異値変換の目的は、特異値を好きな関数へ変換することにある。

\section{量子特異値変換の重要な要素}
量子特異値変換を構成する重要な要素は、以下の3つである。
\begin{itemize}
    \item  量子信号処理(Quantum signal processing)
    \item  量子ビット化(Qubitization)
    \item  ブロック埋め込み(Block encoding)
\end{itemize}
がある。


\section{量子信号処理}
NMRにおける信号強度を上げるための合成パルスに由来

\section{1量子ビットの回転ゲート}
\subsection{パウリ行列}
量子計算に必要となる量子ビットに対する演算子を定義する。
最も重要な演算子に\textbf{パウリ演算子}がある。
\begin{align}
    I = \begin{pmatrix}
    1 & 0 \\
    0 & 1
    \end{pmatrix},
    X = \begin{pmatrix}
    0 & 1 \\
    1 & 0
    \end{pmatrix},
    Y = \begin{pmatrix}
    0 & -i \\
    i & 0
    \end{pmatrix},
    Z = \begin{pmatrix}
    1 & 0 \\
    0 & -1
    \end{pmatrix}
\end{align}

\subsection{回転ゲート}
ブロッホ球上で$x, y, z$軸に$\theta$回転するゲートである。
\begin{align}
    R_A(\theta) &= e^{-i (\theta / 2)A}\\
    &= \cos(\theta / 2)I - i \sin(\theta / 2)A
\end{align}

\section{量子信号処理(Quantum signal processing)}
\subsection{信号演算子}
信号$a$に依存した$x$軸回転である。
\begin{align}
    W(a) = 
    \begin{pmatrix}
    a & i \sqrt{i - a^2} \\
    i \sqrt{1 - a^2} & a
    \end{pmatrix}
    &= R_X(-2 \cos^{-1}(a))
\end{align}

\subsubsection{証明}
以下証明である。
\begin{flalign}
    \begin{pmatrix}
    a & i \sqrt{1 - a^2} \\
    i \sqrt{1 - a^2} & a
    \end{pmatrix} =& a I + i \sqrt{1 - a^2} \\
    =& \cos(\cos^{-1}(a))I + i \sin(\cos^{-1}(a))X  \\
    % (\because \sin(\arccos(a)) = \sqrt{1 - a^2}, \cos(\arccos(a)) = a) \\
    =& \cos(\frac{2 \cos^{-1}(a)}{2})I - i \left(-i \sin(\frac{2 \cos^{-1}(a)}{2})\right)X \\
    =& \cos(\frac{-2\cos^{-1}(a)}{2})I - i \sin(\frac{-2 \cos^{-1}(a)}{2})X \\
    % (\because \sin(- \theta) = - \sin(\theta), \cos(- \theta) = \cos(\theta)) \\
    =& e^{- i (\frac{-2 \cos^{-1}{a}}{2})X} \\
    =& R_X (-2 \cos^{-1}{a})
\end{flalign}

\subsection{信号処理演算子}
信号$a$を処理するために、$z$軸に対する$-2\phi$の回転として、
\begin{align}
    S(\phi) = e^{i \phi Z}
\end{align}
と定義する。

\subsection{量子信号処理操作(QSP)}
これらを$d$回繰り返して定義

量子信号処理は、$x$軸と$z$軸を$d$回、繰り返し作用させる。

% TODO:dについての記述がない

\begin{align}
    U_{\vec{\phi}} = e^{i \phi_0 Z} \prod_{k = 1}^{d} W(a) e^{i \phi_k Z}
\end{align}
$\phi$は上手く人間側が上手く設定することによって、信号$a$を処理する。

\subsection{量子信号処理:一般論}
量子信号処理とは、ある条件を満たす多項式$P(a)$と$Q(a)$に対して、
\begin{align}
    U_{\vec{\phi}} = e^{i\phi_0 Z} \prod^d_{k = 1} W(a)e^{i\phi_kZ}
    = \begin{pmatrix}
    P(a) & iQ(a)\sqrt{1 - a^2} \\
    iQ^*(a)\sqrt{1 - a^2} & P^*(a)
    \end{pmatrix}
    \label{Quantum Signal Processing: General Theory}
\end{align}
を満たすような$\vec{\phi}$が存在する。
$x$軸と$z$軸回転を繰り返した 1量子ビットの回転ゲート。$z$軸回転は好きに決めて良い。

ただし、多項式$P(a), Q(a)$には以下の条件が課せられる。
\begin{itemize}
    \item $P(a)$は$d$次以下、$Q(a)$は$(d - 1)$次以下の多項式
    \item $P(a)$のパリティは$d \mod{2}$、$Q(a)$のパリティは$(d - 1) \mod{2}$
    \item $| P |^2+ (1 - a^2)|Q|^2 = 1$
\end{itemize}
また、$\theta$を適切に選べば、条件を満たす関数であれば、必ず式(\ref{Quantum Signal Processing: General Theory})は成り立つ。

% どういうふうにファイを設定するかを考えておく 

\section{量子ビット化(Qubitization)}
%  1量子ビットのような簡単な系ではなく、
複雑な系でも2次元の部分空間を考えると、あたかも量子ビットの操作として議論できる。

振幅増幅アルゴリズムを見ながら、量子ビット化の考え方を見ていく。

\subsection{振幅増幅アルゴリズム}
あるユニタリ、$U, U^{\dagger}$、および入力$A_{\phi}$と出力$B_\phi$

\begin{align}
    A_{\phi} = e^{i \phi \ket{A_0}\bra{A_0}}, B_{\phi} = e^{i \phi \ket{B_0}\bra{B_0}}
\end{align}
が与えられているときに、

\begin{align}
    |\bra{A_0} Q \ket{B_0}|^2 \rightarrow 1
\end{align}
% A_0を得る確率が限りなく1に近づくようにQを設定する
となる$Q$を構成する。ただし、$\bra{A_0} U \ket{B_0}$は非ゼロとする。
% TODO: 非ゼロとは?

\end{document}