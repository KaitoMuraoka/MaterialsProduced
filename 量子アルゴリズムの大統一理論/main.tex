\RequirePackage{plautopatch}
\RequirePackage[l2tabu, orthodox]{nag}

\documentclass[platex,dvipdfmx]{jlreq}			% for platex
% \documentclass[uplatex,dvipdfmx]{jlreq}		% for uplatex
\usepackage{graphicx}
\usepackage{bxtexlogo}

\usepackage{physics}%ブラケット
\usepackage{amsmath} % 行列表現
\usepackage{amsmath, amssymb}
\usepackage{type1cm}
\usepackage{qcircuit}

\title{レポートタイトル}

\author{学生番号XXX-XXXX アカリク太郎}
\date{\today}
% \begin{document}
% \maketitle
% \section{Cloud LaTeXへようこそ}

% Cloud LaTeXは,\LaTeX を使った文書の作成・管理をクラウド上で行えるWebサービスです.
% \LaTeX を使うと,複雑な数式
% \begin{equation}
% \frac{\pi}{2} =
% \left( \int_{0}^{\infty} \frac{\sin x}{\sqrt{x}} dx \right)^2 =
% \sum_{k=0}^{\infty} \frac{(2k)!}{2^{2k}(k!)^2} \frac{1}{2k+1} =
% \prod_{k=1}^{\infty} \frac{4k^2}{4k^2 - 1}
% \end{equation}
% を含んだ読みやすくきれいな文書作成ができます.

% 本サービスは,\LaTeX 文書をリアルタイムに保存・コンパイルし,ユーザーアカウント別に管理します.
% そのため,本サービスにログインするだけで,どこからでも作業を再開でき,ファイルを持ち歩く必要はありません.
% また,様々な \LaTeX テンプレートが用意されているので,手軽に文書を作り始めることができます.
% \begin{figure}
% \centering
% \includegraphics[width=70mm]{figures/Sample.png}
% \caption{ここにキャプションを挿入します}
% \label{fig:model}
% \end{figure}

% Cloud LaTeXでは,作成されるPDFそのままのレイアウトで表示するPDFビューモードがあり,コンパイル画面を確認しながら文書を作成することができます(図\ref{fig:model})
% 日本語では, \pLaTeX / \upLaTeX / \LuaLaTeX でのコンパイルが可能です.
% また,日本語や英語文書作成だけでなく,中国語・ハングルに対応した \XeLaTeX のコンパイルも可能です.
% ぜひ使ってみてください.
% \end{document}

\begin{document}
\section{量子信号処理}
NMRにおける信号強度を上げるための合成パルスに由来


\subsection{信号演算子}
信号$a$に依存した$x$軸回転である。
\begin{align}
    W(a) = 
    \begin{pmatrix}
    a & i \sqrt{i - a^2} \\
    i \sqrt{1 - a^2} & a
    \end{pmatrix}
    &= R_X(-2 \arccos(a))
\end{align}

\subsubsection{証明}
以下証明である。
% TODO: 計算が合わない
\begin{flalign}
    \begin{pmatrix}
    a i \sqrt{1 - a^2} \\
    i \sqrt{1 - a^2} & a
    \end{pmatrix} =& a I + i \sqrt{1 - a^2} \\
    =& \cos(\arccos(a))I + i \sin(\arccos(a))X  \\
    % (\because \sin(\arccos(a)) = \sqrt{1 - a^2}, \cos(\arccos(a)) = a) \\
    =& \cos(\frac{2 \arccos(a)}{2})I - i \left(-i \sin(\frac{2 \arccos(a)}{2})\right)X \\
    =& \cos(\frac{-2\arccos(a)}{2})I - i \sin(\frac{-2 \arccos(a)}{2})X \\
    % (\because \sin(- \theta) = - \sin(\theta), \cos(- \theta) = \cos(\theta)) \\
    =& e^{- i (\frac{-2 \arccos{a}}{2})X} \\
    =& R_X (-2 \arccos{a})
\end{flalign}

\subsection{信号処理演算子}
$z$軸に対する$-\phi$
\begin{align}
    S(\phi) = e^{i \phi Z}
\end{align}

\subsection{量子信号処理操作(QSP)}
これらを繰り返して定義

量子信号処理は、$x$軸と$z$軸を繰り返し作用させる。

$\phi$は任意である。(人間が調整してやってくださいねぇ〜)

% TODO:dについての記述がない

\begin{align}
    U_{\vec{\phi}} = e^{i \phi_0 Z} \prod_{k = 1}^{d} W(a) e^{i \phi_k Z}
\end{align}

\section{量子信号処理:一般論}
量子信号処理とは、ある条件を満たす多項式$P(a)$と$Q(a)$に対して、
\begin{align}
    U_{\vec{\phi}} = e^{i\phi_0 Z} \prod^d_{k = 1} W(a)e^{i\phi_kZ}
    = \begin{pmatrix}
    P(a) & iQ(a)\sqrt{1 - a^2} \\
    iQ^*(a)\sqrt{1 - a^2} & P^*(a)
    \end{pmatrix}
\end{align}
を満たすような$\vec{\phi}$が存在する。

% どういうふうにファイを設定するかを考えておく 
\end{document}